\section{Compiling}
\subsection{Windows}
\subsubsection{Installing CMake and GNU Make}


To compile this application, you need to have \textbf{CMake}  and \textbf{GNU
Make} on your system.  The easiest way to do this is by using the **scoop
package manager**. **Scoop** is a package manager for windows that automates the
installation of various software using commands. You can learn more about it
here


To do this open powershell run the following command.

\begin{lstlisting}[language=bash]
    iwr -useb get.scoop.sh | iex
\end{lstlisting}

If you get an error you might need to change the execution policy (i.e. enable Powershell), to do so run the following command and redo the command above.

\begin{lstlisting}[language=bash]
    Set-ExecutionPolicy RemoteSigned -scope CurrentUser
\end{lstlisting}

Confirm that you have successfully installed scoop, by running the following command”

\begin{lstlisting}[language=bash]
    scoop --help
\end{lstlisting}

Once you see  you have successfully installed **scoop**.

\begin{lstlisting}[language=bash]
    scoop --help
\end{lstlisting}

\begin{lstlisting}[language=bash]
     Usage: scoop <command> [<args>]

     Some useful commands are:

     alias       Manage scoop aliases
     bucket      Manage Scoop buckets
     cache       Show or clear the download cache
\end{lstlisting}


Now to install \textbf{CMake} and \textbf{GNU Make} run the following command:

\begin{lstlisting}[language=bash]
     scoop install cmake make 
\end{lstlisting}
 
To check whether the installation is successful for \textbf{GNU Make} run the following command:

\begin{lstlisting}[language=bash]
    make --help
\end{lstlisting}

You will be able to see the following:
\begin{lstlisting}[language=bash]
     Usage: make [options] [target] ...
     Options:
       -b, -m                      Ignored for compatibility.
       -B, --always-make           Unconditionally make all targets.
       -C DIRECTORY, --directory=DIRECTORY
                                   Change to DIRECTORY before doing anything
\end{lstlisting}
    
To check whether the installation is successful for **CMake** run the following command:

\begin{lstlisting}[language=bash]
    cmake --help
\end{lstlisting}

You will be able to see the following:

\begin{lstlisting}[language=bash]
Usage

    cmake [options] <path-to-source>
    cmake [options] <path-to-existing-build>
    cmake [options] -S <path-to-source> -B <path-to-build>

Specify a source directory to (re-)generate a build system for it in the
current working directory.  Specify an existing build directory to
re-generate its build system.
\end{lstlisting}


\subsubsection{Compiling the game}

Once you have installed \textbf{CMake} and \textbf{GNU Make}, you may now
compile the application. To do so, navigate to the \codeword{ghostless-pacman} folder and run.
 
\begin{lstlisting}[language=bash]
    make build
\end{lstlisting}

This command will automatically download external GUI libraries which were used
in making the game. After that it will compile the program and place it to the
\codeword{bin} folder.

\subsubsection{Running the game}

Now the only thing left is to run the application, you can navigate to the \codeword{bin}
folder and open the \codeword{ghostless-pacman.exe} file which will run the game. Or you
can run the following command to do this automatically:

\begin{lstlisting}[language=bash]
    make run
\end{lstlisting}

\textbf{NOTE}: It is important to avoid moving the executable since Windows will not
be able to find the DLLs which are required for running the game.


\subsubsection{Running the game}
Use your distribution’s package manager to install the following packages:

% \begin{itemize}
%     \item SDL2
%     \item SDL2_Mixer
%     \item SDL2_Image
% \end{itemize}

Navigate to the the \codeword{ghostless-pacman} folder and run:

\begin{lstlisting}[language=bash]
    make build
\end{lstlisting}