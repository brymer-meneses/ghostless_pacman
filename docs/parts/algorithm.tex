\section{Algorithm}

\subsection{Pre-game Algorithm}

This algorithm initializes once the application runs, in contrast to the
In-game algorithm which runs only when the user chooses Start from the menu,
that is, the actual game.

\begin{enumerate}
    \item Application launches
    \item User chooses from the menu options:
    \begin{enumerate}[label=\alph*.]
        \item
            \begin{itemize}[label={}]
                \item If the user presses 1, the \emph{Start} option will be highlighted
                \item \hspace{1cm} If the user presses Enter, proceed to Step 3
            \end{itemize}
        \item 
            \begin{itemize}[label={}]
                \item If the user presses 2, the \emph{Tutorial} option will be highlighted
                \item \hspace{1cm} the user presses Enter, proceed to Step 4
            \end{itemize}
        \item 
            \begin{itemize}[label={}]
                \item  If the user presses 3, the \emph{Exit} option will be highlighted
                \item \hspace{1cm} If the user presses Enter, proceed to Step 5
            \end{itemize}
        \item 
            \begin{itemize}[label={}]
                \item  If the user presses A, the \emph{About the Game} option will be highlighted
                \item \hspace{1cm} If the user presses Enter, proceed to Step 3
            \end{itemize}
        
        \item Else, display a wrong input reminder.
    \end{enumerate}
    
    \item User chooses the number of food pieces s/he wants for the game
        \begin{enumerate}[label=\alph*]
            \item If the user input is from 2 to 9, display chosen number
            \begin{itemize}[label={}]
                \item If the user presses Enter, proceed to In-game Algorithm
                \item Else, display a wrong input reminder
            \end{itemize}
            \item If the user presses M, go back to Step 2
            \item Else, display a wrong input reminder
        \end{enumerate}
    \item Program shows the first slide of the Tutorial
        \begin{enumerate}[label=\alph*]
        \item User navigates through the slides by pressing arrow left (←) and arrow right (→)
        \item If the user presses M, go back to Step 2
        \item If the user presses 1, proceed to Step 3
        \item If the user presses X, proceed to Step 5
        \item Else, display a wrong input reminder
        \end{enumerate}
    \item Program prompts a quit window
        \begin{enumerate}[label=\alph*]
            \item If the user presses Y, exit the program
            \item If the user presses N, the prompt closes
        \end{enumerate}
\end{enumerate}


\subsection{In-game Algorithm}

As mentioned, this algorithm initializes when the user chooses Start from the menu and actually play the game.

\begin{enumerate}
    \item Display a 10-by-10 board, with Pacman on the top left, the food
        pieces based on the accepted user input, the randomly distributed
        blocks, and the randomly placed exit door
        \begin{enumerate}[label=\alph*]
            \item If the user presses W, pacman moves up
            \item If the user presses S, pacman moves down
            \item If the user presses A, pacman moves left
            \item If the user presses D, pacman moves right
            \item If the user presses M, the program goes back to menu (go back to Step 2 of the Pre-game Algorithm)
            \item If the user presses X, the program prompts a quit window (recall Step 5 of the Pre-game Algorithm)
            \begin{itemize}[label={}]
                \item If the user presses Y, exit the program
                \item If the user presses N, the prompt closes
            \end{itemize}
        \end{enumerate}
    \item User plays the game by moving Pacman to eat the food pieces towards the exit door
        \begin{enumerate}[label=\alph*]
            \item If Pacman hits a block, display a Game Over prompt
            \item If Pacman gets out of the board, display a Game Over prompt
            \item If Pacman reaches the door without eating all the food pieces, display a Game Over prompt
            \item If Pacman reaches the door after eating all the food pieces, display a Game Victory prompt
        \end{enumerate}
    \item With the prompts from the game results, the user chooses between the options to restart, return to menu, or exit the game
        \begin{enumerate}[label=\alph*]
            \item If the user presses R, restart the game (go back to Step 3 of the Pre-game Algorithm)
            \item If the user presses M, return to menu (go back to Step 2 of the Pre-game Algorithm)
            \item If the user presses X, the program prompts a quit window (recall Step 5 of the Pre-game Algorithm)
                \begin{itemize}[label={}]
                    \item If the user presses Y, exit the program
                    \item If the user presses N, the prompt closes
                \end{itemize}
            \item Else, display a wrong input reminder
        \end{enumerate}
\end{enumerate}
